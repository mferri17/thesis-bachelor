%\cleardoublepage
%\begingroup
%\let\clearpage\endgroup
\null\vspace{\stretch{1}}


\chapter*{\centering Abstract}
\addcontentsline{toc}{chapter}{Abstract}

Lo studio nasce dalla curiosità di comprendere i meccanismi che regolano il funzionamento di un motore di ricerca. Lo scopo finale consiste nella progettazione e realizzazione di un sistema che, a partire da un'importante quantità di informazioni estratte da fonti eterogenee, permetta agli utenti di effettuare, in maniera intuitiva, ricerche rapide ed efficienti per ottenere risultati in linea con le aspettative.

\vspace{1\baselineskip}

La tesi si sviluppa secondo un approccio di indagine progressivo. 

Nella prima parte dello studio viene analizzato il contesto all’interno del quale nasce l’esigenza di effettuare una ricerca, quali sono le problematiche che ne scaturiscono e le soluzioni usualmente adottate. Particolare attenzione è dedicata al concetto di analisi del testo, presentato attraverso modelli e algoritmi che formalizzano il problema.

L’approccio teorico è seguito da un capitolo dedicato ad Apache Solr, piattaforma di \textit{enterprise search} che svincola l’utilizzatore dai modelli matematici e implementa ad alto livello i concetti precedentemente affrontati. Dopo una breve occhiata al funzionamento di Apache Solr, vengono presentate le soluzioni che esso propone per affrontare lo sviluppo di un sistema basato sulla ricerca \textit{full text}.

La terza parte dell’elaborato costituisce quella più corposa. Viene preso in considerazione un caso di studio reale, un software di tipo \textit{Customer Relationship Management} (CRM) con un orientamento \textit{Applicant Tracking System} (ATS) per la Ricerca e la Selezione del Personale. All’analisi dei requisiti segue la progettazione del motore di ricerca, successivamente sviluppato con il supporto di Solr ed un approccio di miglioramento incrementale. In questa fase vengono discussi i problemi riscontrati e le scelte intraprese per la loro risoluzione.

Come risultato dello stage viene infine presentato il prototipo del motore di ricerca da me sviluppato a supporto ed integrazione del software già esistente. Particolare attenzione è dedicata all’esperienza utente, parte integrante della progettazione del sistema sia da un punto di vista funzionale che prestazionale.


\vspace{\stretch{2}} \null
