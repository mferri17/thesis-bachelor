
\chapter*{Introduzione}
\addcontentsline{toc}{chapter}{Introduzione}

Si è recentemente diffuso il concetto di “big data” e al giorno d'oggi sono costantemente a disposizione in rete un’enorme quantità di dati, i quali rappresentano tutto ciò che l’uomo conosce, produce e di cui fa uso nella propria quotidianità. Fino al secolo scorso era difficile immaginare che la rivoluzione tecnologica a cui si è assistito negli ultimi anni avrebbe generato un incremento così massiccio dei dati che vengono perennemente scambiati da una parte all’altra del mondo. Tale quantità è destinata ad aumentare ed è di fondamentale importanza che si impari a gestire il fenomeno, affinché questi dati non siano solo un insieme di lettere e numeri ma vengano interpretati perché possano costituire un elemento informativo.

\vspace{1em}

Il primo responsabile dell’aumento esponenziale dei dati è stato il World Wide Web, grazie al quale si è vista nascere la possibilità di condividere con chiunque il proprio sapere e creare così un’immensa rete di conoscenza e di informazioni interconnesse. Insieme alla nascita del WWW, negli anni ’90, si diffonde quindi la necessità di doversi orientare nella miriade di pagine e documenti che hanno popolato la rete di anno in anno, ed è a partire da questa esigenza che i motori di ricerca hanno assunto un ruolo fondamentale. Se in precedenza erano utilizzati solo in ambiti aziendali ed istituzionali, i motori di ricerca sul Web si sono progressivamente fatti conoscere ad un pubblico decisamente più ampio e tutt’oggi sono costantemente utilizzati da chiunque.

Con il passare del tempo, i motori di ricerca sono diventati sempre più precisi e hanno iniziato ad essere utilizzati per consultare e ricercare oltre che informazioni testuali, anche contenuti multimediali: foto, audio e video. Se la ricerca di foto e video è stata in prima battuta accompagnata dalla catalogazione manuale - cioè umana - delle informazioni, negli ultimi anni si sta assistendo all’automatizzazione del processo, per merito di sistemi di riconoscimento sempre più precisi e diffusi in grado di determinare automaticamente quale possa essere il contenuto di un’immagine o di un video. Infine, nonostante i motori di ricerca di tracce audio siano sicuramente meno conosciuti, chiunque di noi ne ha probabilmente utilizzato uno senza sapere che si trattasse proprio di questo: Shazam. Il medesimo tipo di ricerca consente inoltre di rilevare in maniera automatica contenuti multimediali sul Web in violazione dei diritti d’autore.

\vspace{1em}

Appena un decennio dopo la sua comparsa, il WWW ha portato con sé anche l’avvento dei \textit{social media}, che hanno determinato un cambiamento radicale nelle vite di tutti. Tale cambiamento è causa dell’elevata diffusione di nuovi contenuti digitali e informazioni personali che, ogni giorno, vengono numerosamente condivisi con la propria rete di contatti. Il 2018 è stato e continua ad essere un anno di fervore proprio su questo argomento ed in particolare sulla condivisione, detenzione e utilizzo dei dati personali. 

Se quel che facciamo, come ci comportiamo, viene reso disponibile online da noi stessi attraverso un \textit{social network} o indirettamente attraverso il nostri spostamenti (si pensi all’utilizzo sempre più diffuso dei \textit{wearable device} o delle automobili \textit{smart}), è chiaro che tali informazioni devono essere raccolte ed elaborate da qualcuno. Questo “qualcuno” non sono altro che le aziende proprietarie delle tecnologie e delle piattaforme che quotidianamente utilizziamo per condividere le nostre informazioni, ed è essenziale che tali aziende sappiano gestire questi dati affinché possano costituire un elemento informativo, che generi un profitto o quantomeno si riveli di una certa utilità.

\vspace{1\baselineskip}

In un mondo nel quale il “dato” sembra essere l'elemento sotto l’attenzione di tutti, che le grandi aziende dei \textit{social} sanno opportunamente gestire per ottenere dei risultati profittevoli, si assiste contemporaneamente all’esistenza di numerose realtà all’interno delle quali non si riesce a valorizzare a dovere i dati che si detengono, che talvolta risultano sprecati o di difficile consultazione. Accade spesso, infatti, che molte aziende o istituzioni siano responsabili di enormi archivi documentali ricchi di informazioni potenzialmente utili, che possono però finire per rivelarsi più un peso che un’opportunità.

Questo elaborato nasce proprio dall’esigenza di fare ordine all’interno di un archivio documentale e, in particolare, considera un caso di studio reale per il settore delle risorse umane. È importante notare, infatti, che la stragrande maggioranza dei processi di ricerca e selezione del personale si svolgono attraverso il solo scambio di un documento di testo: il \textit{curriculum vitae} (CV). Nonostante si assista a molte situazioni che prevedono l’utilizzo di appositi moduli di candidatura, atti a strutturare precisamente le informazioni fornite dal candidato, anche questi si concludono sempre e comunque con la richiesta di un CV. 

Essendo il \textit{curriculum} un file testuale, è chiaro che un archivio composto da questo tipo di documenti appare complicato da consultare e ciò determina una seria difficoltà nel reperire in esso informazioni rilevanti in maniera rapida ed efficace. Per questo motivo, si è cercato di comprendere in che modo sia possibile elaborare archivi costituiti da documenti testuali affinché vi si possano facilmente reperire delle informazioni che si rivelino effettivamente utili agli occhi di un \textit{recruiter}.

\vspace{1em}

Lo studio condotto sui temi di reperimento delle informazioni ha prodotto come risultato il prototipo di un motore di ricerca, il cui funzionamento è stato integrato all’interno di Arkivium Recruiting, software adibito alla gestione dei processi di ricerca e selezione del personale. Questo progetto di laurea documenta le fasi che si sono susseguite per la progettazione e lo sviluppo del motore di ricerca, che ha l’obiettivo di consentire all’utente di cercare - attraverso un’unica interfaccia Web - all’interno della totalità delle informazioni a propria disposizione, cioè in maniera congiunta all’interno di conoscenze strutturate, provenienti da un database, ed in quelle non strutturate (o semi-strutturate) contenute nei \textit{curriculum} o in altri tipi di file testuali.
















